\documentclass{book}
\usepackage{bbm}
\usepackage{fullpage}

\renewcommand*\descriptionlabel[1]{\hspace\leftmargin$#1$}

\title{Quantitative Proprietary Trading}
\author{Ray Ng}
\begin{document}
\maketitle
\tableofcontents


\chapter{Volatility targeting}
\label{ch:voltgt}

Chapter~\ref{ch:voltgt} is about volatility targeting and about sizing trades for a given risk. 


\section{Unit Risk}
\label{sn: unitrisk}
The inputs would be a percentage risk tolerance and outputs is simply number of units of the asset to but to obtain that risk. First we define funding risk per unit of instrument as:

\begin{equation}
\label{eq: unitrisk}
\sigma_{\mathbbm{1},i} = \sigma_{\%,i}X_{t,i}fx_{t,i}
\end{equation}
where:
\begin{description}
\item[\sigma_{\%,i}] is annual percentage volatility
\item[X_{t,i}] is asset price
\item[fx_{t,i}] is fx rate from asset price numeraire to funding numeraire example jpy to usd.
\end{description}
Note that estimating the annual percentage volatility can be a separate model on its own but using either ewm std or simple std is typically used. Taking the inverse of Eq.~\ref{eq: unitrisk}, we get:
\begin{equation}
\mathbbm{N}_{\mathbbm{1}} = \frac{1}{\sigma_{\mathbbm{1},i}}
\end{equation}
where
\begin{description}
\item[\mathbbm{N}_{\mathbbm{1}}] is the number of units to purchase to get \$1 of risk.
\end{description}

\section{Volatility targeting in a portfolio}
\label{sn: pfvoltgt}
The point of documenting this is for figuring out how to trade a proper multi factor portfolio.

\subsection{Risk Weights}
\label{ssn: riskweights}
This is how to implement it mechanically:

\begin{enumerate}

\item{Obtain AUM for the portfolio: $V(t)$}

\item{Obtain portfolio target volatility: $\sigma_{target}$ and obtain cash volatility target. This is hard coded. Compute daily cash vol:
\begin{equation}
\sigma_{target,daily}= \sigma_{target}V(t) / 16 
\end{equation}
}
%\item{Calculate portfolio cash risk weights by multiplying by predefined handcrafted risk weights. A starting point can always be equal risk weights. I call this base weights

\begin{equation}
\vec{W}_{base, \$ } = \vec{w}_{base, i}   \sigma_{target,daily}
\end{equation}

\item{Now convert to instrument units:

\begin{equation}
\vec{N}_{base} = \vec{W}_{base, \$} \mathbbm{N}_{\mathbbm{1}}
\end{equation}

\item{Finally adjust it using signals. We assume the usual -20 to 20 signal scale:
\begin{equation}
\vec{N}_{\alpha} = \vec{N}_{base} S/ 20
\end{equation}
}
\end{enumerate}


\subsection{Cash Weights}
\label{ssn: cashweights}
We can apply the same methodology to cash weights with the simple modification of not requiring to use unit risks.


\begin{enumerate}

\item{Obtain AUM for the portfolio: $V(t)$}

\item{We assume we are given handcrafted cash weights: $W_{\%}$}

\item{We can obtain instrument level cash allocations: 

\begin{equation}
W_{\$,i} = W_{\%} V(t)
\end{equation}

}


\item{Now convert to instrument units so that:

\begin{equation}
N_{\%,i} =  W_{\$,i} / X_i
\end{equation}
}

\end{enumerate}

Note that the difference in both protocols comes down to the definition of the risk conversion numeraire. It is $\sigma_{\mathbbm{I},i}$ if using risk weights but simply the price: $X_t$ if using cash weights weighting. Trading any portfolio would only require knowledge of the AUM to invest in it. The same also applies for a multifactor portfolio. I would also like to use this to figure out how to trade various rebalancing schedules as a means for diversification.




\chapter{Systematic backtesting}
\label{ch:sysbt}

Chapter~\ref{ch:sysbt} is about the implementation of a generic systematic backtester. Note that at the highest level, the system should basically spit out the trades required for the day. This would be a difference between the target portfolio and the current portfolio. Short of an automated trading strategy, the trader is expected to execute faithfully.  

\section{Data layer}
Here's the vectorized data require and that can be precalculated:

\begin{description}
\item[X_i(t)] $\equiv$ price data at fixed frequency
\item[\sigma_{\mathbbm{I}}] $\equiv$ is the unit risk, meaning how many units to buy per \$1 of risk, where the funding units is arbitrary. 
\item[fx_{i}(t)] $\equiv$ is the fx conversion rate from asset price numeraire to funding numeraire example JPY to USD.
\item[\Delta PL_{i}(t) \equiv \left(X_{i}(t+1)-X_i(t)\right)] is the daily PL in [DOM] units for longing one unit of $i$
\item[\Delta PL_{i,\$}(t) \equiv \Delta PL_{i}(t)fx_{i}(t)] is the daily PL in [\$] units for longing one unit of $i$
\item[\Delta PL_{i,C}(t) \equiv \left( r_{FOR}- r_{DOM} \right)\delta t] is the carry earned for longing one unit of $i$
\item[\vec{\delta X}(t) \equiv \vec{X}^{bid}(t) -  \vec{X}^{ask}(t)] is the bid ask spread for the portfolio per unit asset traded]
\item[\vec{\Delta PL}_{tcost}(t) = \vec{N}_{rebalance}(t)\circ \vec{\delta X}(t)/2 \circ \vec{fx}(t).]
\end{description}

\section{Portfolio backtest}
Assuming we had a time series of target inventory: $\vec{N}(t)$, then we can compute the daily PL of the portfolio without transaction costs:
\begin{equation}
\vec{\Delta PL}(t) = \vec{N}(t-1) \circ \vec{\Delta PL (t)} 
\end{equation}
and the portfolio PL is simply given by:
\begin{equation}
\Delta PL (t) = \vec{\Delta PL}(t) \circ \vec{\mathbbm{1}}
\end{equation}
If we wish to include transaction costs, we need to assume rebalance flow. 
\begin{equation}
\vec{N}_{rebalance}(t) = \vec{N(t-1)} - \vec{N(t)}.
\end{equation}
then the rebalance transaction cost can be estimated as:
\begin{equation}
\vec{\Delta PL}_{tcost}(t) = \vec{N}_{rebalance}(t)\circ \vec{\delta X}(t)/2 \circ \vec{fx}(t).
\end{equation}
The full daily PL is then given by:
\begin{equation}
\vec{\Delta PL}_{tcost}(t) = \vec{N}_{rebalance}(t)\circ \vec{\delta X}(t)/2 \circ \vec{fx}(t).
\end{equation}







\end{document}